\documentclass[fleqn,10pt]{wlscirep}
\usepackage[utf8]{inputenc}
\usepackage[T1]{fontenc}
\usepackage{lineno}

\linenumbers

\title{Microelectrode electrophysiology: Extending the Brain Imaging Data Structure to intracellular and extracellular recordings in animal models}

% Add authors
\author[1,*]{First Author}
% Add co-authors from INCF Working Group, tool developers, early adopters

\begin{abstract}
% 170 words max, no references
% - BIDS has facilitated data sharing in human neuroimaging
% - We present extension for microelectrode electrophysiology in animal models
% - Two datatypes: icephys and ecephys
% - Three-level hierarchy: probes, electrodes, channels
% - Dual coordinate system: probe-relative and anatomical
% - NWB/NIX as data formats
% - Example datasets and tool integration (SpikeInterface, neuroconv)
% - Enables reproducible analysis, data sharing via DANDI/G-Node/EBRAINS
The Brain Imaging Data Structure (BIDS) has facilitated data sharing and tool development in human neuroimaging.
We present an extension for microelectrode electrophysiology recordings in animal models, addressing the unique requirements of intracellular and extracellular recordings.
This extension introduces two new data types: `icephys' for intracellular and `ecephys' for extracellular recordings, supporting diverse recording modalities from patch-clamp to high-density silicon probes.
Building on existing BIDS principles and prior electrophysiology extensions, we specify metadata for probes, electrodes, and channels, with particular attention to metadata required for spike sorting analysis.
The extension adopts NWB (Neurodata Without Borders) and NIX (Neuroscience Information Exchange) as data formats, ensuring comprehensive metadata capture while maintaining compatibility with existing analysis ecosystems.
We provide example datasets covering common use cases and demonstrate integration with established tools including [Which tools?].
This standardization enables reproducible analysis pipelines, facilitates data sharing through repositories like DANDI, G-Node and EBRAINS, and bridges scales from cellular to systems neuroscience.
\end{abstract}

\begin{document}

\flushbottom
\maketitle

\thispagestyle{empty}

\section*{Background \& Summary}

% (700 words maximum) An overview of the study design, the assay(s) performed, and the created data, including any background information needed to put this study in the context of previous work and the literature. The section should also briefly outline the broader goals that motivated the creation of this dataset and the potential reuse value. We also encourage authors to include a figure that provides a schematic overview of the study and assay(s) design. The Background \& Summary should not include subheadings. This section and the other main body sections of the manuscript should include citations to the literature as needed.

% 700 words max

% Paragraph 1: Field context and technological advances
Microelectrode electrophysiology encompasses techniques for recording electrical activity from individual neurons to local field potentials, providing crucial insights into neural computation.
Recent technological advances, including high-density silicon probes and standardized probe designs, have dramatically increased data acquisition rates and experimental complexity.
% Add statistics on data growth, number of labs using these techniques

% Paragraph 2: Current standardization landscape and fragmentation
While comprehensive data formats exist for neurophysiology (NWB; NIX),
the field lacks standardized organization principles for datasets, metadata specifications, and directory structures.
This fragmentation impedes data sharing, with surveys indicating [ADD SURVEY DATA] of researchers struggling to share or reuse electrophysiology data due to inconsistent formats and missing metadata.
% Reference Neo, existing tools showing fragmentation

% Paragraph 3: BIDS success and extension rationale
BIDS has successfully standardized human neuroimaging data organization [cite],
with over 850 datasets on OpenNeuro [cite] and adoption by major repositories.
Prior BIDS extensions for human electrophysiology (EEG [cite], MEG [cite], iEEG [cite])
established patterns for organizing time-series neural data, while the Microscopy extension [cite]
introduced critical metadata fields for animal data.

% Paragraph 4: Unique challenges for microelectrophysiology
% New content specific to BEP032
Microelectrode recordings present unique challenges:
(1) electrode scales spanning orders of magnitude (sub-micron tips to millimeter arrays),
(2) diverse probe geometries requiring specialized coordinate systems,
(3) spike sorting as an essential preprocessing step requires specific metadata,

% Paragraph 5: Extension overview and contributions
Here we present BEP032, extending BIDS to microelectrode electrophysiology, with a focus on animal models.
This extension: [summarize key contributions]
% List main contributions

\section*{Microephys-BIDS Summary}

% Brief overview pointing to Figure 1

% Directory structure
% - sub-<label>/[ses-<label>]/ecephys/ or icephys/
% - Data files: *_ecephys.nwb or *_ecephys.nix
% - Sidecar JSON: *_ecephys.json
% - Three TSV files: probes.tsv, electrodes.tsv, channels.tsv
% - Optional: space-<label>_electrodes.tsv + coordsystem.json

% Data formats
% - NWB (.nwb) - widely adopted
% - NIX (.nix) - HDF5 backend, Neo interface
% - Rationale: open, comprehensive metadata, existing ecosystems

% Metadata inheritance
% - JSON/TSV at dataset, subject, or session level
% - Lower levels override higher
% - Reduces redundancy

\section*{Specific Microephys-BIDS Considerations}

\subsection*{Probes, Electrodes, and Channels}

% CRITICAL distinction (similar to iEEG-BIDS)
% - Probe: physical device (Neuropixels, tetrode bundle, patch pipette)
% - Electrode: individual recording site on a probe
% - Channel: recorded signal, may combine multiple electrodes

% Why three-level hierarchy (vs iEEG's two-level)?
% - Complex probe geometries (hundreds of sites)
% - Multiple probes per recording common
% - Probe metadata distinct from electrode positions

% File contents:
% - probes.tsv: probe_name, type, manufacturer, AP, ML, DV, angles, hemisphere
% - electrodes.tsv: name, probe_name, x, y, z (probe-relative)
% - channels.tsv: name, type, units, sampling_frequency, reference, electrode linkage

The iEEG-BIDS and EEG-BIDS extensions distinguished electrodes from channels: an electrode is a contact point with tissue, a channel is the amplifier and digitizer that produces a stored time series. 
These are not equivalent, a differential recording derives one channel from two electrodes, a Neuropixels electrode may yield separate highpass and LFP channels at different sampling rates, and a sync signal is a channel with no electrode at all. 
The prior extensions noted that electrodes can be ``organized into arrays, grids, or probes,'' but their metadata captured only electrodes and channels, not this grouping.

We add an explicit probe level. 
A probe is the physical device, Neuropixels, tetrode bundle, patch pipette, carrying one or more electrodes. This matters because probe-level metadata (manufacturer, surgical coordinates, insertion angle) applies to all electrodes on that probe, while electrode-level metadata (position, impedance) defines the geometry, and channel-level metadata (gain, filters) describes the acquisition.

Three files capture this. \texttt{probes.tsv} lists each probe with its type and surgical placement. 
\texttt{electrodes.tsv} enumerates contact sites, linking each to its probe and specifying probe-relative coordinates for spike sorting. 
\texttt{channels.tsv} documents recorded signals: sampling frequency, units, source electrode, and reference. 
Channels without electrodes (sync, stimulus) use \texttt{n/a}; ground, typically hardware like a skull screw or ear clip, is defined globally in the sidecar JSON and can be overridden per-channel if it differs. 
A \texttt{stream\_id} column links channels to data objects within the NWB or NIX file.

\subsection*{Intracellular vs Extracellular Datatypes}

% Why two datatypes?
% - Fundamentally different techniques
% - Different signal characteristics (mV vs μV, single cell vs population)
% - Different metadata (pipette_solution, recording_mode for icephys)
% - Aligns with NWB terminology

% icephys: patch-clamp, sharp electrode
% ecephys: silicon probes, tetrodes, microwire arrays, Utah arrays

\subsection*{Coordinate Systems}

% KEY DIFFERENCE from iEEG-BIDS
% - iEEG: always anatomical space
% - Microephys: dual approach

% Without space-<label>:
% - x, y, z in electrodes.tsv are probe-relative
% - Origin at probe tip
% - Essential for spike sorting

% With space-<label>:
% - Anatomical/atlas-based coordinates
% - Requires coordsystem.json
% - Supported systems:
%   Rodent: AllenCCFv3, PaxinosWatson, FranklinPaxinos, WaxholmSpace
%   Primate: CHARM, D99, PaxinosMacaque, MarmosetBrainAtlas
%   Generic: Stereotaxic, individual, Other

% Surgical coordinates in probes.tsv:
% - AP, ML, DV relative to Bregma/Lambda
% - AP_angle, ML_angle, rotation_angle
% - Sign conventions in specification appendix

\subsection*{Data Format Requirements}

% Why require NWB/NIX (vs native formats)?
% - Native formats too heterogeneous (dozens of manufacturers)
% - NWB/NIX already widely adopted
% - HDF5-based, open, self-documenting
% - Conversion tools exist (neuroconv, Neo)

% stream_id column in channels.tsv
% - Links channels to data objects in NWB/NIX
% - Handles multiple streams (LFP, spike band)

\subsection*{ProbeInterface Integration}

% Community probe geometry library
% - model column in probes.tsv with TermURL
% - References ProbeInterface definitions
% - Automated electrode position lookup
% - Examples: Neuropixels 1.0, NeuroNexus A1x32-Poly3

\section*{Software and Tools}

% Tool support:
% - neuroconv: native formats to NWB/NIX
% - SpikeInterface: spike sorting with BIDS I/O
% - BIDS validator: schema-based validation
% - Example scripts at [GITHUB]

\section*{Data Records}

% Example datasets at [REPOSITORY]:
% 1. Neuropixels in mouse visual cortex (ecephys)
% 2. Patch-clamp in mouse cortex (icephys)
% 3. Chronic tetrodes in rat hippocampus
% Brief descriptions, species, DOIs

\section*{Technical Validation}

% Validation approach:
% - BIDS validator with microephys schema
% - Round-trip tests: native -> NWB -> BIDS -> analysis
% - Cross-tool compatibility (SpikeInterface, Neo)

\section*{Usage Notes}

% Adoption guidance:
% - Full spec at bids-specification.readthedocs.io
% - Workflow: native -> neuroconv -> NWB/NIX -> BIDS structure
% - Repository submission: DANDI, G-Node, EBRAINS

\section*{Code Availability}

% - Specification: https://github.com/bids-standard/bids-specification
% - Example datasets and scripts: [GITHUB LINK]

\bibliography{bep032}

\section*{Acknowledgements}
% INCF Working Group members, funders, community contributors

\section*{Author Contributions}
% CRediT-style contributions

\section*{Competing Interests}
The authors declare no competing interests.

\section*{Figures \& Tables}

\begin{figure}[ht]
\centering
% \includegraphics[width=\linewidth]{figures/bep032_overview}
\caption{Overview of the BIDS microelectrode electrophysiology extension showing an example extracellular electrophysiology dataset.
Left: directory structure with the \texttt{ecephys/} datatype folder containing data files and metadata.
Right: file contents showing (a) the sidecar JSON with recording metadata, (b) \texttt{probes.tsv} describing probe placement and type, (c) \texttt{electrodes.tsv} with probe-relative electrode positions, (d) \texttt{channels.tsv} linking recorded signals to electrodes, and (e) example voltage traces.
Numbered labels connect files in the directory tree to their content panels.}
\label{fig:overview}
\end{figure}

\end{document}
